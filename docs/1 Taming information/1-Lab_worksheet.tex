% Options for packages loaded elsewhere
\PassOptionsToPackage{unicode}{hyperref}
\PassOptionsToPackage{hyphens}{url}
%
\documentclass[
]{article}
\usepackage{amsmath,amssymb}
\usepackage{lmodern}
\usepackage{iftex}
\ifPDFTeX
  \usepackage[T1]{fontenc}
  \usepackage[utf8]{inputenc}
  \usepackage{textcomp} % provide euro and other symbols
\else % if luatex or xetex
  \usepackage{unicode-math}
  \defaultfontfeatures{Scale=MatchLowercase}
  \defaultfontfeatures[\rmfamily]{Ligatures=TeX,Scale=1}
\fi
% Use upquote if available, for straight quotes in verbatim environments
\IfFileExists{upquote.sty}{\usepackage{upquote}}{}
\IfFileExists{microtype.sty}{% use microtype if available
  \usepackage[]{microtype}
  \UseMicrotypeSet[protrusion]{basicmath} % disable protrusion for tt fonts
}{}
\makeatletter
\@ifundefined{KOMAClassName}{% if non-KOMA class
  \IfFileExists{parskip.sty}{%
    \usepackage{parskip}
  }{% else
    \setlength{\parindent}{0pt}
    \setlength{\parskip}{6pt plus 2pt minus 1pt}}
}{% if KOMA class
  \KOMAoptions{parskip=half}}
\makeatother
\usepackage{xcolor}
\usepackage[margin=1in]{geometry}
\usepackage{graphicx}
\makeatletter
\def\maxwidth{\ifdim\Gin@nat@width>\linewidth\linewidth\else\Gin@nat@width\fi}
\def\maxheight{\ifdim\Gin@nat@height>\textheight\textheight\else\Gin@nat@height\fi}
\makeatother
% Scale images if necessary, so that they will not overflow the page
% margins by default, and it is still possible to overwrite the defaults
% using explicit options in \includegraphics[width, height, ...]{}
\setkeys{Gin}{width=\maxwidth,height=\maxheight,keepaspectratio}
% Set default figure placement to htbp
\makeatletter
\def\fps@figure{htbp}
\makeatother
\setlength{\emergencystretch}{3em} % prevent overfull lines
\providecommand{\tightlist}{%
  \setlength{\itemsep}{0pt}\setlength{\parskip}{0pt}}
\setcounter{secnumdepth}{-\maxdimen} % remove section numbering
\ifLuaTeX
  \usepackage{selnolig}  % disable illegal ligatures
\fi
\IfFileExists{bookmark.sty}{\usepackage{bookmark}}{\usepackage{hyperref}}
\IfFileExists{xurl.sty}{\usepackage{xurl}}{} % add URL line breaks if available
\urlstyle{same} % disable monospaced font for URLs
\hypersetup{
  pdftitle={1 (TW5) Lab worksheet},
  pdfauthor={Dr.~Chris Moreh},
  hidelinks,
  pdfcreator={LaTeX via pandoc}}

\title{1 (TW5) Lab worksheet}
\author{Dr.~Chris Moreh}
\date{}

\begin{document}
\maketitle

We begin our exploration of sociological research methods by looking at
some real data that social scientists have used in empirical research.
Before starting to think in a more structured way about research
questions, how to design a sociological research project, and how to
create useful instruments to capture and collect relevant information
about the social world, let's look at some `semi-tamed' information:
data collected by others for academic and/or policy analysis purposes.

\hypertarget{set-readings}{%
\subsection{Set readings}\label{set-readings}}

There are no set readings for this week, just the general assigned core
and secondary readings.

\hypertarget{exercise-1-exploring-the-uk-data-service-ukds}{%
\subsection{Exercise 1: Exploring the UK Data Service
(UKDS)}\label{exercise-1-exploring-the-uk-data-service-ukds}}

The UK Data Service is the country's largest data repository. It makes a
variety of research data available for UK researchers and students, and
we will be using secondary data accessible through this repository a lot
in this module (and for the assignments!). It is also useful to become
acquainted with the variety of secondary data available there and the
basic functionalities of the site, as you may decide to use secondary
data for your final third-year dissertation.

The UKDS is available at this website:
\url{https://ukdataservice.ac.uk/}

\hypertarget{task-1-browse-data-by-type}{%
\subsubsection{Task 1: Browse data by
type}\label{task-1-browse-data-by-type}}

Begin by exploring the options available under the \textbf{Find data}
tab of the UKDS website. The \textbf{Find data} page contains a short
video on \emph{How to use the UK Data Service catalogue search tool},
which you can watch outside class. As we are just exploring the data
offerings of the service, navigate to the \emph{Browse and access data}
page, where you can browse data by theme or type, among some other
options.

Let's look into the first option: \emph{UK Surveys}. This takes us to
the main data catalogue, where we can see that the search is filtered
down to ``UK Survey data'' in the \emph{Data Type} field of the menu on
the left. We can perform various selections using that menu, refining
the search by date or setting other filters.

\begin{quote}
❓ Question:
\end{quote}

\hypertarget{task-2}{%
\subsubsection{Task 2:}\label{task-2}}

\hypertarget{task-3}{%
\subsubsection{Task 3:}\label{task-3}}

\hypertarget{exercise-2-explore-some-cross-national-surveys}{%
\subsection{Exercise 2: Explore some cross-national
surveys}\label{exercise-2-explore-some-cross-national-surveys}}

\end{document}
